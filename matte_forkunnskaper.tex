% Created 2023-01-29 sø. 16:53
% Intended LaTeX compiler: pdflatex
\documentclass[11pt, A4paper]{article}
\usepackage[left=1in,top=.5in]{geometry}%[textheight=26cm,textwidth=17cm]
\usepackage[utf8]{inputenc}
\usepackage[T1]{fontenc}
\usepackage{graphicx}
\usepackage{longtable}
\usepackage{wrapfig}
\usepackage{rotating}
\usepackage[normalem]{ulem}
\usepackage{amsmath}
\usepackage{amssymb}
\usepackage{capt-of}
\usepackage{natbib}
\usepackage[norsk]{babel}
\usepackage{physics}
\usepackage{amsmath,booktabs}
\raggedright 
\setlength{\parskip}{1em} % larsm

\author{Lars}
\date{\today}
\title{}

\begin{document}

\begin{equation*}
\def\arraystretch{2.5}
\begin{array}{c@{\hspace{1cm}}ll@{}}
%\toprule
     \multicolumn{2}{c@{}}{\text{Matte-repetisjon. Først noen oppgaver (2-5 minutter etter litt øvelse). Svar neste side.}} \\
  %\cmidrule(l){2-3}
  \hline
    F=ma                & a=                   &  \\
    PV=nRT              & \frac{n}{V}=         &\\
    C=\frac{n}{V}       & V=  &\\
    \cdot               & 2^3\times2^{-4}=       &  \\
     y=Ae^x             & x=  &  \\
     y=Ae^{-kx}          & x=   &  \\
     y=Ae^{-\frac{E}{kT}}  & E=   & \\
     y=\ln{x}           & x=   & \\
     \cdot              & \ln{e^{x}}=      & \\
  (\mathrm{anta\,\, } x>0)                 & e^{\ln{x}}= &    \\
           \cdot        & e^{a}e^{b} = &\\
           \cdot        & \ln{x^{3}}-3\ln{x} =  &\\
              \cdot     & \ln({AB/C}) - (\ln{A} + \ln{B} - \ln{C}) =  &\\
         \cdot          & \dv{}{t}\left(Ae^{-kt}\right) = &\\
  \mathrm{(konstant\, y)}& \pdv{}{x}\left(x^2y+\tan({\ln{y}})\right) = & \\
  \dv{x^n}{x} = nx^{n-1}}& \dv{}{x}\left(\frac{1}{x}\right) = &\\
  (a>0\,, b > 0)   &\int_a^b1/x^{2}\,dx =&\\
  (a>0\,, b > 0)   &\int_a^b1/x \,dx =&\\
\end{array}
\end{equation*}


\begin{equation*}
\def\arraystretch{2.5}
\begin{array}{c@{\hspace{1cm}}ll@{}}
%\toprule
     \multicolumn{2}{c@{}}{\text{Noen mulige svar \hspace{30em}}} \\
     % \cmidrule(l){2-3}
  \hline
    F=ma                & a=F/m                   &  \\
    PV=nRT  & \frac{n}{V}=\frac{P}{RT} & \\
    C=\frac{n}{V}       & V=\frac{n}{C}  &\\
    \cdot                    & 2^3\times2^{-4}=2^{3-4}=2^{-1}=1/2       &  \\
     y=Ae^x             & x=\ln(y/A)  &  \\
     y=Ae^{-kx}          & x=-\ln(y/A)/k \,\,\,\,\,\, (\mathrm{eller\,\,}  \ln(A/y)/k)   &  \\
     y=Ae^{-\frac{E}{kT}}  & E=-kT\ln(y/A)   & \\
     y=\ln{x}           & x=e^y   & \\
  \cdot                      & \ln{e^{x}}=x      & \\
  (\mathrm{anta\,\, } x>0)                 & e^{\ln{x}}=x &    \\
       \cdot                 & e^{a}e^{b} = e^{a+b}&\\
        \cdot                & \ln{x^{3}}-3\ln{x} = 0 &\\
        \cdot                & \ln({AB/C}) - (\ln{A} + \ln{B} - \ln{C}) = 0 &\\
       \cdot                 & \dv{}{t}\left(Ae^{-kt}\right) = -Ake^{-kt}&\\
  \mathrm{(konstant\, y)}& \pdv{}{x}\left(x^2y+\tan({\ln{y}})\right) = 2xy& \\
  \dv{x^n}{x} = nx^{n-1}}& \dv{}{x}\left(\frac{1}{x}\right) = \dv{x^{-1}}{x} = -x^{-2}=-\frac{1}{x^2}&\\
 (a>0\,, b > 0)   &\int_a^b1/x^{2}\,dx =\left[-\frac{1}{x}\right]_{a}^{b} = -(\frac{1}{b}-\frac{1}{a}) = \frac{1}{a} - \frac{1}{b} &\\
 (a>0\,, b > 0)   &\int_a^b1/x \,dx = \left[\ln(x)\right]_{a}^b = \ln(b)-\ln(a) = \ln(b/a)&\\

\end{array}
\end{equation*}



\section*{Matte}

Her er litt uformell repetisjon av ting det vil være en stor fordel å kunne godt. Skriv ned noen ganger det du har glemt.

\section{Logaritmer}

Per definisjon er $\ln(x)$, den naturlige logaritmen av x,  det tallet $e$ må opphøyes i for å gi x.

\begin{equation*}
\label{eq:6}
x = e^{\ln(x)}
\end{equation*}

Så logaritmen til et tall \textit{er eksponenten}, dersom tallet skrives som $e$ opphøyd i en eksponent.

Så hvis (og bare hvis)
\begin{equation*}
\label{eq:8}
x = e^y
\end{equation*}
har vi at
\begin{equation*}
\label{eq:9}
y = \ln(x)
\end{equation*}
Eksponent-funksjonen og logaritmen er invers-funksjoner (``undo''-funksjoner) for hverandre:\newline $e^{\ln(x)}=x$ og $\ln(e^x)=x$. 

Vi må også kunne at
\begin{equation*}
\label{eq:10}
e^{a+b} = e^ae^b
\end{equation*}
så
\begin{equation*}
\label{eq:12}
e^{b+\ln(a)} = e^b e^{\ln{a}} = ae^{b}
\end{equation*}

Av samme grunn gjør logaritmen mulitiplikasjon til addisjon, og divisjon til subtraksjon.
\begin{equation*}
\label{eq:7}
\ln(a b) = \ln(a) + \ln(b)
\end{equation*}
\begin{equation*}
\label{eq:13}
\ln{\frac{a}{b}} = \ln{a}-\ln{b}
\end{equation*}

Relevant oppgave: Ta logaritmen på begge sider av:
\begin{equation*}
\label{eq:14}
k = A e^{-\frac{E_a}{RT}}
\end{equation*}
og
\begin{equation*}
\label{eq:15}
C = C_0 e^{-kt}
\end{equation*}

Lag noen oppgaver selv hvis dette er uklart, eller test med tall i Python, R, regneark
eller på kalkulatoren.  Husk at det er forskjell på $\log_{10}$ og
$\ln$ (som er $\log_e$). I de fleste programmeringsspråk kalles $\ln$ $\log$.

\section{Partiellderiverte}
Det er mye partiellderivering i fysikalsk kjemi. Her er en enkel og uformell måte å forstå partiellderiverte og vanlige deriverte på. 

$dx$ betyr ``en bitte liten endring'' \footnote{``... a little bit of x``; Thompson, Calculus Made Easy. https://www.gutenberg.org/files/33283/33283-pdf.pdf. Egentlig er jo den deriverte definert som en grenseverdi, men det går som regel fint å tenke på infinitesimale dx etc.} av x.

$\frac{dy}{dx}$ er den deriverte, en liten endring av y delt på en
liten endring av x. Her er y en funksjon av x, slik at hvis x endres
litt så endres også y litt \footnote{med mindre den deriverte er null}, og så deler man endringen av y på endringen av x. 

Hvis vi nå tenker oss at energien $U$ til et system (for eksempel en flaske farris
eller en ballong) er en funksjon av temperaturen $T$, ønsker vi kanskje å vite
hvor raskt energien stiger når temperaturen stiger. Så da kan
det være interessant å studere
\begin{equation*}
\label{eq:16}
\frac{\mathrm{d}U}{\mathrm{d}T}\,.
\end{equation*}
Men hvis $U$ også er avhengig av andre variable som ikke nødvendigvis endrer seg når $T$ endrer seg, da må man ha en mer presis notasjon. Derfor  angir man helt
eksplisitt hvilke variable som holdes konstant. Man bytter ut $d$ med
$\partial$ og skriver f.eks.
\begin{equation*}
\label{eq:1}
\left(\frac{\partial U}{\partial T}\right)_{P,n}
\end{equation*}
Dette kalles partiellderivering, men er ikke vanskelig. Det betyr bare at $P$ og $n$ holdes konstant. Ellers er det fortsatt bare
``en bitte liten endring av $U$ delt på en (tilsvarende) bitte liten
endring av $T$``, gitt at $U$ kan beskrives som som en funksjon av
$T$, $P$ og $n$ (antall mol). Om du velger å tenke på endringen som en
faktisk liten fysisk endring i et begerglass, eller om du bare tenker
på egenskapene til matematiske variabler og funksjoner, er opp til
deg.

(Når det sies at $U$ er en funksjon av $T$ og $P$
menes det forresten ikke nødvendigvis at denne funksjonen er et matematisk uttrykk
som kan skrives ned på papir, bare at $U$ er gitt dersom $T$ og $P$
er gitt for systemet.)

Vær klar over at samme symbol ofte brukes både om en fysisk størrelse
og om funksjonen for størrelsen, som når det står $U = U(T, V)$. $U$
brukes da både for indre energi og for en funksjon for indre
energi. Fra matematikken er dere kanskje vant med $z = f(x, y)$. Noen
likninger dere vil se er ikke fysiske lover, bare beviste matematiske
regler, av typen

\begin{equation}
\label{eq:3}
dz = \left(\pdv{z}{x}\right)_y dx + \left(\pdv{z}{y}\right)_{x} dy\,.
\end{equation}

(sml likning 3E.2 på side 104 Atkins 11. utgave eller 3D.2 s 140 10. utgave)

\section{Dobbeltderiverte}
\begin{equation}
\label{eq:2}
\dv[2]{f(x)}{x} = \dv{}{x}\dv{f(x)}{x}
\end{equation}

Eksempel:
\begin{equation}
\label{eq:4}
\dv[2]{}{x}\left(x^{3}\right) = \dv{}{x}\left(3x^2\right) = 6x
\end{equation}
For partiellderiverte gjelder følgende regel\footnote{
  Hvis du lurer på hvorfor:
  \begin{equation*}
\pdv{y}(\pdv{f(x,y)}{x}) = \lim_{\Delta y\to 0}\frac{
  \lim_{\Delta x\to 0}
  \frac{f(x+\Delta x, y+\Delta y) - f(x, y+\Delta y)}{\Delta x}
  - \lim_{\Delta x\to 0}\frac{f(x+\Delta x, y) - f(x, y)}{\Delta x}}{\Delta y}
\right).
\end{equation*}
Dette blir det samme som  
\begin{equation*}
  \pdv{x}(\pdv{f(x,y)}{y})= \lim_{\Delta x\to 0}
  \frac{
  \lim_{\Delta y\to 0}
  \frac{f(x+\Delta x, y+\Delta y) - f(x+\Delta x, y)}{\Delta y}
  - \lim_{\Delta y\to 0}\frac{f(x, y+\Delta y) - f(x, y)}{\Delta y}}{\Delta x}
\right).
\end{equation*}
}:
Det er det samme om du deriverer først med hensyn på x og deretter med hensyn på y eller først med hensyn på y og deretter med hensyn på x.
\begin{equation}
\label{eq:5}
\pdv[2]{f(x,y)}{x}{y} = \pdv{}{x}\left(\pdv{f(x,y)}{y}\right) = \pdv{}{y}\left(\pdv{f(x,y)}{x}\right)
\end{equation}
Dvs egentlig bør jeg skrive:
\begin{equation}
\label{eq:11}
\pdv[2]{f(x,y)}{x}{y} = \left(\pdv{\left(\pdv{f(x,y)}{y}\right)_x}{x}\right)_y =
\left(\pdv{\left(\pdv{f(x,y)}{x}\right)_y}{y}\right)_x
\end{equation}
\begin{figure}[h]
  \centerline{\includegraphics[width=.7\textwidth]{deriverte.png}}
  \caption{Deriverte og andrederiverte. Positiv derivert betyr at kurven stiger, positiv andrederivert betyr at den deriverte stiger.}
  \label{fig:interp_example}
\end{figure}

\newpage
\section{Integraler}
Et integral er en sum over mange bittesmå ledd.

\begin{figure}[h]
  \centerline{\includegraphics[width=.65\textwidth]{integral.png}}
  \caption{Illustrasjon av integral, her representert som et areal A. Prøv å forstå at $\dv{A}{x} = f(x).$} (Egentlig skal $dx$ gå mot null).
  \label{fig:interp_example}
\end{figure}


Vi vet at ``arbeid er lik kraft ganger strekning'', og måles i Joule (J). 1J = 1Nm (Newton meter). Eksempel: En vekt på 1.02 kg veier 10N; for å løfte denne vekten 10 meter opp må det gjøres et arbeid på 10N$\times$10m = 100J. Så arbeidet $W=Fs$, hvor $F$ er kraften og $s$ strekningen. Men hvis vi løfter vekten noen hundre km opp, er ikke lenger $F$ konstant, og da må vi integrere:
\begin{equation*}
\label{eq:17}
W=\int_0^s F(x) dx\,.
\end{equation*}

(Dette er også relevant for krefter mellom molekyler og for gasser som komprimeres eller utvides.)


  
\begin{figure}[h]
  \centerline{\includegraphics[width=.9\textwidth]{work_sum.png}}
  \caption{Arbeid (energi) er kraft ganger strekning, summert.}
  \label{fig:interp_example}
\end{figure}

(Ett viktig resultat: Med $F=ma$ kan vi med $\int Fdx$ regne ut endring av kinetisk energi: 
%\begin{equation*}
%\label{eq:18}
$W = \int F dx = \int ma dx = \int m \dv{v}{t} dx =\int m \dv{v}{t} v dt = \int m v dv = \frac{1}{2} m v^2 + C$)
%\end{equation*}


\end{document}